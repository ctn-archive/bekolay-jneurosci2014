\documentclass[11pt]{article}
\usepackage{amsmath,amsfonts,amsthm,amssymb}
\usepackage{setspace}
\usepackage{fancyhdr}
\usepackage{lastpage}
\usepackage{extramarks}
\usepackage{chngpage}
\usepackage{caption}
\usepackage{namedplus}
\usepackage[usenames,dvipsnames]{color}
\usepackage{graphicx,float}
\usepackage{listings}
\usepackage{framed}
\usepackage{inconsolata}
\usepackage{hyperref}
\usepackage{booktabs}
\hypersetup{
  colorlinks = true, %Colours links instead of ugly boxes
  urlcolor = blue, %Colour for external hyperlinks
  linkcolor = blue, %Colour of internal links
  citecolor = blue %Colour of citations
}

% Stuff to change, you know, if you want.
\setlength{\parindent}{1cm}
\setlength{\parskip}{3pt}

% In case you need to adjust margins:
\topmargin=-0.45in      %
\evensidemargin=0in     %
\oddsidemargin=0in      %
\textwidth=6.5in        %
\textheight=9.0in       %
\headsep=0.25in         %

\pagestyle{plain}

\setlength\fboxrule{1pt}
\linespread{1.5}

\graphicspath{{../figures/}}

%%%%%%%%%%%%%%%%%%%%%%%%%%%%%%%%%%%%%%%%%%%%%%%%%%%%%%%%%%%%%
% Make title
\title{A spiking neural integrator model of the adaptive control of
  action by the medial prefrontal cortex}
\date{Abbreviated title: A spiking model of mPFC adaptive control}
\author{%
  Trevor Bekolay \\
  Centre for Theoretical Neuroscience \\
  University of Waterloo \\ ~ \\
  Mark Laubach \\
  The John B. Pierce Lab and Dept. of Neurobiology \\
  Yale University School of Medicine \\ ~ \\
  Chris Eliasmith \\
  Centre for Theoretical Neuroscience \\
  University of Waterloo}
%%%%%%%%%%%%%%%%%%%%%%%%%%%%%%%%%%%%%%%%%%%%%%%%%%%%%%%%%%%%%
\begin{document}

\maketitle

\textbf{Corresponding author:} Chris Eliasmith.
Email: \url{celiasmith@uwaterloo.ca}.
Centre for Theoretical Neuroscience, University of Waterloo.
200 University Avenue West, Waterloo, ON, N2L 3G1.

\textbf{Pages:} 27. \textbf{Figures:} 10. \textbf{Tables:} 2. \textbf{Multimedia, 3D models:} 0

\textbf{Word counts:} Abstract: 199, Introduction: 486, Discussion: 1499.

\textbf{Conflict of Interest}
The authors declare no competing financial interests.

\textbf{Acknowledgements}

Financial support: NSF 1121147 to ML;
NSERC Discovery, CRC, CFI, and OIT to CE;
and NSERC CGS-D to TB.
We thank Benjamine Liu for technical contributions
to the modeling effort and
Nandakumar Narayanan who recorded the neurons
that were analyzed by TB and ML.

\clearpage

\begin{abstract}
  Subjects performing simple reaction time tasks
  can improve reaction times by learning the expected timing
  of action-imperative stimuli
  and preparing movements in advance.
  Success or failure on the previous trial
  is often an important factor for determining
  whether a subject will attempt to time
  the stimulus or wait for it to occur
  before initiating action.
  The medial prefrontal cortex (mPFC) has been implicated
  in enabling the top-down control of action
  depending on the outcome of the previous trial.
  Analysis of spike activity from the rat mPFC
  suggests that neural integration is a key mechanism
  for adaptive control in precisely timed tasks.
  We show through simulation
  that a spiking neural network
  consisting of coupled neural integrators
  captures the neural dynamics
  of the experimentally recorded mPFC.
  Errors lead to deviations in the normal
  dynamics of the system,
  a process that could enable learning
  from past mistakes.
  We expand on this coupled integrator network
  to construct a spiking neural network that
  performs a reaction time task by following
  either a cue-response or timing strategy,
  and show that it performs the task
  with similar reaction times as experimental subjects
  while maintaining the same spiking dynamics
  as the experimentally recorded mPFC.
\end{abstract}

\linespread{1.5}

\section{Introduction}

The medial prefrontal cortex (mPFC) has been implicated in regulating reinforcement learning parameters
\cite{Amiez2005,Khamassi2012},
conflict monitoring
\cite{Botvinick2004,VanVeen2004,Aarts2009,Sheth2012},
performance monitoring
\cite{Carter1998,Fecteau2003,Rushworth2004,Brown2005,Modirrousta2008,Histed2009,Alexander2011,Horst2012,Hyman2013},
anticipatory control \cite{Koyama2001,Aarts2008},
arousal of the sympathetic nervous system
\cite{Critchley2003,Luu2003},
and control of action timing
\cite{Muir1996,Mulert2003,Risterucci2003,Narayanan2009,Naito2013}.
Conceptual models have been proposed
that would allow the mPFC to be involved
in several of these functions,
depending on behavioral context
(e.g., \citenoparens{Luu2003,Botvinick2004}).
Mathematical models
based on reinforcement learning
explain experimental results
at several levels
(e.g., \citenoparens{Alexander2011,Khamassi2012}).
However, these reinforcement learning models
have not been shown to explain spiking activity
of neural ensembles recorded during
temporally constrained behavioral tasks.
Here, we present a dynamical spiking neural model
that relates directly to
both neural and behavioral data
in a simple mPFC-dependent task.

Specifically, we propose a
two-dimensional dynamical system
that can account for mPFC activity
related to error monitoring and action timing
in rodents performing a simple reaction time task.
Neural recordings in the motor cortex
during the acquisition of this task
reveal that an encoding of forthcoming errors
develops with learning \cite{Laubach2000}.
These activities depend on processing in the mPFC
\cite{Narayanan2006Neuron},
where neurons encode errors
both prospectively and retrospectively
\cite{Narayanan2006Neuron,Narayanan2008}.
Analyses of population activity during the task
show that both the mPFC and motor cortex
are dominated by
relatively slow fluctuations in firing rates,
starting from the initiation of the trial
and terminating when feedback is given
about the trial's outcome
\cite{Narayanan2009}.

Here, we show that a specific type of network structure,
called a double integrator network \cite{Singh2006},
can play a central role in
accounting for reaction time task performance.
With analyses identical
to the experimental study,
we show that the neural activity produced
by the simulated spiking network closely matches
the experimental data from the
\citetext{Narayanan2009} study.
Furthermore, we propose a feedforward control system
that can perform the simple reaction time task
and show that the double integrator network
can be used to modify the behavior
of the control system based on
the outcome of the previous trial.
When the control system is implemented
in a spiking neural network,
the modified behavior closely matches
the behavior of experimental subjects
performing the same reaction time task.

\section{Methods}

\subsection{Reaction time task} \label{sec:simple-rt}

The experimental data analyzed in this study is the same
as that in \citetext{Narayanan2009},
where more detailed explanation of the experimental
methods can be found.
In that study, twelve male Long-Evans rats
were trained to press down a lever
for a fixed amount of time (the ``foreperiod'';
1 second for this study)
using standard methods \cite{Laubach2000}.
After the foreperiod, an auditory cue is presented.
The subject then has 0.6 seconds (the response window)
in which to release the lever.
If the lever is released in that time,
the trial is a success, and the subject is rewarded
with water.
If the lever is released during the foreperiod,
the trial is classified as a ``premature error,''
and is penalized by a time-out with the house lights extinguished.
If the lever is not released within the response window,
the trial is classified as a ``late error,''
and is also penalized by a time-out with the house lights extinguished.
Figure~1A presents a schematic view of the task.

Rats included in the study reached performance criterion
of $> 60 \%$ correct trials in 9.75 $\pm$ 1.5 sessions
(1748 $\pm$ 343 trials).
Microwire electrode arrays were then implanted
into the prelimbic region of the rat cerebral cortex,
which is generally considered to be
part of the medial prefrontal cortex
\cite{Laubach2011} and the anterior cingulate cortex
in particular \cite{Medalla2009}.
Data were recorded using
a Plexon Many Neuron Acquisition Program
for 3 sessions (567 $\pm$ 98 trials).

\subsubsection{Simulation}

In order to provide appropriate
environmental input for
the networks described
in subsequent sections,
the same reaction time task is simulated
using a finite state machine.
The task can be in one of seven states~($S$):
trial start~(TS), foreperiod~(FP),
timeout~(TO), cue~(C), response window~(R),
reward~(Rw) and intertrial interval~(ITI).
State transitions can be seen in
the graphical depiction of
the finite state machine in Figure~1C.

The auditory cue, house lights, and reward signals
are provided by the environment
based on the current task state.
\begin{equation} \label{eq:env}
  u_C =
  \begin{cases}
    1 &\text{if } S = \text{C} \\
    0 &\text{otherwise}
  \end{cases}
  \quad
  u_{TO} =
  \begin{cases}
    1 &\text{if } S = \text{TO} \\
    0 &\text{otherwise}
  \end{cases}
  \quad
  u_{Rw} =
  \begin{cases}
    1 &\text{if } S = \text{Rw} \\
    0 &\text{otherwise}
  \end{cases}
\end{equation}
The lever position does not depend on task state,
but instead on press ($u$) and release ($u_r$)
signals provided by a control system
(see section~\ref{sec:control}).
\begin{equation} \label{eq:lever}
  L = \int [-u(t) + u_r(t)] \, dt
\end{equation}

\subsection{Double integrator network} \label{sec:dint}

A key finding of the \citetext{Narayanan2009}
study was that the temporal profiles
of the leading principal component
from the mPFC resembled the cumulative sum
of the second component, and vice versa.
This finding suggested that population activity
in the mPFC might reflect a process of integration.
Using the Nengo simulator \cite{BekolayInpress},
a number of candidate networks
were investigated and one particular network
was found to resemble the dynamics
of the mPFC recordings.
This network is called a ``double integrator network,''
and was first proposed in the context of working memory
by \citetext{Singh2006}.
We hypothesize that the double integrator network
proposed by \citeauthor{Singh2006}
will exhibit the same neural dynamics
as the experimental mPFC
when modified to respond
to events occurring during the simple RT task.

In order to test this hypothesis,
we extend the double integrator network
by providing it with multiple inputs
that represent events occurring during
the simple RT task
(including those in Equation~[\ref{eq:env}]),
while maintaining the same underlying dynamics.
By driving the double integrator network
to certain parts of the state space,
we make it possible to predict
the state of the system
when the cue occurs during the
reaction time task.

\subsubsection{Dynamics} \label{sec:dynamics}

The underlying dynamics of the double integrator network
are captured in the following equations:
\begin{align*}
  \dot{x}_1 &= u \\
  \dot{x}_2 &= \beta x_1,
\end{align*}
where $u$ represents an input signal
and $\beta$ is a scalar parameter representing the strength of
the connection between the two integrators.
This results in a two-dimensional system
in which $x_1$ integrates its input,
maintaining the accumulated value over time, and
$x_2$ integrates the value of $x_1$.
This is the system proposed by \citetext{Singh2006},
who predicted that this dynamical system is capable
of time tracking;
because $x_2$ integrates $x_1$,
the amount of time elapsed since
$x_1$ changes can be determined
based on how much $x_2$ has changed.

In \citetext{Singh2006}, neural recordings
were analyzed during a single trial
of the vibrotactile discrimination task.
This corresponds to a single trial
of the simple RT task
in which the input, $u$,
is the lever press,
and the initial state, $x$, is $(0, 0)$.
However, in the simple RT task we perform
many consecutive trials.
We therefore use other inputs from
the environment to match
the trajectories tracked by
the experimental principal components
(shown in Figure~2).

Since the principal components
resemble the double integrator network
in the post-correct case,
we aim to reset the system to the origin point
after correct trials.
Figure~2D indicates that
the principal components return to their
original state after approximately 2 seconds,
which is the time at which
the subject receives reward.
Therefore, we hypothesize that
the trial's outcome (i.e., reward delivery
or a timeout) resets the system
to some particular state,
as proposed in \citetext{Narayanan2009}.
In the case of reward delivery,
we implement this hypothesis
by using the reward delivery signal, $u_{Rw}$,
to drive the system toward $(0, 0)$.
\begin{align*}
  \dot{x}_1 &= u - R \, u_{Rw} \, x_1 \\
  \dot{x}_2 &= \beta x_1 - R \, u_{Rw} \, x_2,
\end{align*}
where $R$ is a scalar parameter
indicating how strongly the system
is driven to the origin point.

In the post-error case,
the principal components
in Figure~2D are
not at the origin point.
Since the lever press increases $x_1$,
when $x_1 > 0$ we can conclude that
the subject is currently performing the task.
We therefore hypothesize that $x_1 < 0$
indicates that an error has recently occurred.
We implement this hypothesis
by using the house lights signal,
$u_{TO}$, to decrease $x_1$.
\begin{equation} \label{eq:dint}
  \begin{aligned}
    \dot{x}_1 &= u - R \, u_{Rw} \, x_1 - E u_{TO} \\
    \dot{x}_2 &= \beta x_1 - R \, u_{Rw} \, x_2,
  \end{aligned}
\end{equation}
where $E$ is a scalar parameter
controlling how quickly $x_1$ decreases.
Note that $x_2$ is not directly driven by $u_{TO}$;
however, because of the $\beta x_1$ term,
$x_2$ will also decrease.

Finally, Figure~2D shows that
the principal components are unstable
during the intertrial interval.
To model trial-by-trial cycles of
excitability associated with task engagement,
a slow oscillation ($\sim0.2$ Hz)
was applied to the networks.
The final dynamics of the double integrator are
\begin{equation} \label{eq:dint-o}
  \begin{aligned}
    \dot{x}_1 &= u - R \, u_{Rw} \, x_1 - E u_{TO} + \sin(0.2 \cdot 2 \pi) \\
    \dot{x}_2 &= \beta x_1 - R \, u_{Rw} \, x_2,
  \end{aligned}
\end{equation}
where $\beta$, $R$, and $E$ are positive scalars.
The effects of changing these free parameters
are explored in Section~\ref{sec:params}.

\subsection{Control networks} \label{sec:control}

In order to evaluate the double integrator network's
ability to influence action timing,
we created a feedforward control network
that performs the simple RT task
by releasing the lever
when the cue arrives;
we refer to this feedforward system as the cue-responding network.
We connected the double integrator network
to the cue-responding network such that
the lever can also be released
by the double integrator network;
we refer to this combined system as the adaptive control network.

\subsubsection{Cue-responding network} \label{sec:cue-responding}

The cue-responding network
mimics how an experimental subject
would react to the cue
if the time of the cue cannot be predicted.
The trial starts after
the intertrial interval ends ($t_s$),
at which point the lever is pressed.
\begin{equation*}
  u(t) =
  \begin{cases}
    1 &\text{if } t > t_s \text{ and } L > -1 \\
    0 &\text{otherwise}.
  \end{cases}
\end{equation*}

The lever is released
once the cue occurs at $t_c$.
\begin{equation} \label{eq:cue-responding}
  u_r(t) =
  \begin{cases}
    1 &\text{if } t > t_c \text{ and } L < 1\\
    0 &\text{otherwise}
  \end{cases}
\end{equation}

\subsubsection{Adaptive control network}

There is a fixed lower bound
on reaction times in the cue-responding network
based on the time taken to detect the cue
and for $u_r$ to integrate to 1;
with direct simulation of Equation~\eqref{eq:cue-responding}
there is no delay between cue onset
and cue detection,
but in a spiking neural implementation
there are signal transmission delays
between each neural connection.
Since the goal is to release the lever
within the response window,
the cue-responding strategy
may be too slow if there are too many intervening
connections between
the neural population detecting the cue,
and the neural population driving
the muscles that eventually
effect lever release.

The double integrator network described
previously can be combined with the
cue-responding network to implement
a faster strategy because
the state of the dynamical system being tracked by
the double integrator (Equation [\ref{eq:dint-o}])
is approximately the same on each post-correct trial.
If the previous trial was a correct trial,
then the double integrator's state
approaches (1, 1) as the time of the cue approaches,
for some values of $\beta$.
We therefore add an additional term
that effects lever release
if the predicted cue time is approaching.
\begin{equation} \label{eq:release}
  u_r = u_r + \frac{1}{1 + e^{-a (x_2 - b)}}
\end{equation}
The new term defines a sigmoid that increases
from 0 to 1 when $x_2 \approx b$.
The slope of the smooth increase
from 0 to 1 is controlled by $a$;
for very large $a$, the sigmoid
is essentially a step function.
We choose $a = 20$ and $b = 0.9$,
which corresponds to a sigmoid
that starts increasing
when $x_2 = 0.6$ and reaches 1 when $x_2 = 1.2$.

If the previous trial was an error trial,
the double integrator
will follow a different trajectory
in which $1 / 1 + e^{-a (x_2 - b)} \approx 0$
throughout the trial,
and therefore the control system
will only release the lever
once the cue occurs.
This allows the adaptive control network
to adopt an aggressive prediction strategy
if the last trial was successful,
and to adopt a conservative cue-response strategy
if the last trial was not successful.

% It is likely that learning
% takes place after error trials
% in order to perform better
% on subsequent trials.
% Incorporating this in the adaptive control network
% would be possible by modifying $\beta$
% when errors occur;
% however, that extension to the adaptive control network
% is outside the scope of this paper.

\subsection{Neural Engineering Framework} \label{sec:nef}

The previous sections describe
a dynamical system hypothesis
of the mPFC and surrounding areas
of cortex that together
perform the simple reaction time task
adaptively depending on the outcome
of the previous trial.
While these can be numerically simulated
to observe ideal system behavior,
we construct a spiking neural network
implementing the above equations
in order to test the hypothesis
in a biologically plausible setting
that approximates the direct numerical simulation.
Testing this neural implementation
allows us to determine if the approximation
is able to perform the task
in a manner comparable to the real brain
by producing simulated neural data
that can be compared directly with
experimental neural data.
We construct spiking neural networks
using the principles of the Neural Engineering Framework
(NEF; \citenoparens{Eliasmith2003})
in the Nengo simulation environment
\cite{BekolayInpress}.
Using the NEF, each signal
from the dynamical systems above
are represented with a population of spiking neurons,
and transformed through the connections
between those populations
(see Figure~3).

The NEF enables the simulation of dynamical systems
by making the assumption that populations of spiking neurons
represent real-valued vectors,
and using a least-squares optimal method
to compute connection weights that
transform those representations
through the connections between
populations of neurons.
A population can have recurrent connections
in order to produce dynamics,
such as the integrative populations
employed in this study.

The NEF's representation scheme
is a kind of population coding
\cite{Georgopoulos1986,Salinas1994},
extended to $n$-dimensional vector spaces.
Each neuron in a population is sensitive
to a particular direction,
called the neuron's \textit{encoder}, denoted $\mathbf{e}$.
The activity of a neuron can be expressed as
\begin{equation} \label{eq:activity}
  a = G[J(\mathbf{x})], \quad
  J(\mathbf{x}) = \alpha \mathbf{e} \cdot \mathbf{x} + J_{bias} + J_\eta,
\end{equation}
where $G[\cdot]$ is a spiking nonlinear neural activation function,
$\alpha$ is a scaling factor (gain) associated with the neuron,
$\mathbf{e}$ is the neuron's encoder,
$\mathbf{x}$ is the vector to be encoded,
$J_{bias}$ is the background current of the cell
when $\mathbf{x} = 0$,
and $J_\eta$ is noise current injected
to match experimental neural variability.
The encoded value, $\mathbf{x}$,
can be estimated linearly:
\begin{equation*}
  \mathbf{\hat{x}}(t) = \sum_i \mathbf{d}_i a_i(t),
\end{equation*}
where $a_i$ is the activity of neuron $i$
and $\mathbf{d}_i$ is a decoding weight
determined by solving a least-squares minimization
of the difference between the decoded estimate
and the actual encoded vector.

The least-squares minimization is solved
with the following equations.
$\mathbf{X}$ is a set of samples from
the representational range of
$\mathbf{x}$ (e.g., samples from $\mathcal{U}(-1, 1)$
in a typical scalar case).
\begin{align*}
  \mathbf{A} &= \begin{bmatrix}
                 a_0(\mathbf{X}) \\
                 a_1(\mathbf{X}) \\
                 \vdots \\
                 a_n(\mathbf{X})
               \end{bmatrix} \\
  \mathbf{d} &= \mathbf{\Gamma^{-1} \Upsilon}, \quad \text{where }
  \mathbf{\Gamma} = \mathbf{AA^T} \quad \text{and } \mathbf{\Upsilon} = \mathbf{AX^T}
\end{align*}

Neural activity is interpreted as a
filtered spike train.
\begin{equation*}
  a_i(t) = \sum_s h(t - t_s) = \sum_s e^{-(t - t_s) / \tau_{PSC}},
\end{equation*}
where $h(\cdot)$ is an exponential filter
modeling postsynaptic current that is
applied to each spike,
and $s$ is the set of all spikes occurring
before the current time $t$.

Connection weights implementing a linear transform
of the encoded vectors of a population
can be computed as
\begin{equation*}
  \omega_{ij} = \alpha_j \mathbf{e}_j L \mathbf{d}_i,
\end{equation*}
where $i$ indexes the input population,
$j$ indexes the output population,
and $L$ is a linear operator.
Nonlinear transformations can be computed
by solving for decoders that minimize
the difference between the decoded estimate and
a function of the encoded vector;
i.e., $\mathbf{\Upsilon} = \mathbf{A} f(\mathbf{X^T})$.

Dynamical systems of the form
$\dot{\mathbf{x}} = A(\mathbf{x}) + B(\mathbf{u})$
can be implemented by connecting a population
to itself with the linear operator
$L = A$ on the recurrent connection,
and receiving input $\mathbf{u}$
from other populations and
setting $L = B$ on those connections.

The above principles are not dependent
on a particular neuron model, $G[\cdot]$.
In this study, we use the
adaptive leaky integrate-and-fire (ALIF) model
\cite{Koch1999} to match \citetext{Singh2006}.
This model is governed by the equations:
\begin{align*}
  \frac{dV}{dt} &= - \frac{V (1 + RG_{adapt}) - J(\mathbf{x})
    R}{\tau^{RC}} \\
  \frac{dG_{adapt}}{dt} &= - \frac{G_{adapt}}{\tau_{adapt}} \\
  \text{if } V > V_{th}&, \:\: G_{adapt} = G_{adapt} + G_{inc},
\end{align*}
where $R$ is the leak resistance,
$\tau^{RC}$ is the RC time constant,
and $G_{adapt}$ tracks how much
the conductance is modulated over time.
$G_{inc}$ and $\tau_{adapt}$
are parameters affecting $G_{adapt}$.

For each instance of each spiking neural network,
the parameters of each ALIF neuron
are randomly selected from
a biologically plausible range
(e.g., the maximum firing rate,
$\mathcal{U}(10\text{Hz}, 50\text{Hz})$,
produces average firing rates
that match \citenoparens{Narayanan2009}).
The encoders, $\mathbf{e}_i$, are
randomly selected from
a uniform distribution of directions
in the encoded vector space.
Table~\ref{tab:nef-param} summarizes
the parameters specific
to the neural implementation.
All other parameters are described
in Sections~\ref{sec:dint} and \ref{sec:control}.

\begin{table}
  \begin{center}
    \begin{tabular}{cccccc}
      \toprule
      Maximum firing rate & $J_\eta$ & $\tau^{RC}$ & $\tau_{ref}$ &
        $\tau_{adapt}$ & $G_{inc}$ \\
      \midrule
      $\mathcal{U}(10\text{ Hz}, 50\text{ Hz})$ &
      $\mathcal{U}(-0.2\text{ nA}, 0.2\text{ nA})$ & 20ms & 1ms & 10ms &
        $\mathcal{U}(0.001, 0.02)$ \\
      \bottomrule
    \end{tabular}
  \end{center}
  \caption{Parameters used in the neural implementation
    of the double integrator and control networks.
    Maximum firing rate and $J_\eta$ are ensemble-level parameters
    as they apply to any neuron model $G[\cdot]$.
    The maximum firing rate is used to determined the gain ($\alpha$)
    and bias currents ($J_{bias}$) associated with neuron
    (and used in Equation~[\ref{eq:activity}]).
    All other parameters apply specifically to the ALIF neuron model.}
  \label{tab:nef-param}
\end{table}

\subsection{Data analysis}

Both experimental and simulated experiments
produced data at the neural and behavioral levels.
The analyses below were done
on both experimental and simulated data,
unless otherwise noted.

\subsubsection{Neural data}

Experimental spike trains from Long-Evans rats
were determined by online identification
with an oscilloscope and audio monitor,
and offline spike sorting
using Plexon software.
Spike sorting was based on principal component analysis
(PCA; described below) and waveform shape.
In all, 174 single units were identified
in rodent mPFC and analyzed together
\cite{Narayanan2009}.

Simulated spike trains were generated
by the Nengo simulator.
The populations representing signals
in the double integrator and control
networks (see Sections~\ref{sec:dint} and \ref{sec:control})
contained 1200 adaptive leaky integrate-and-fire neurons
(see Figure~3);
174 of the neurons in an analyzed population
were randomly selected
in order to match the number of units
from the experimental study.

Spike trains are analyzed
with principal component analysis (PCA),
following procedures used in \citetext{Narayanan2009}
to characterize common firing patterns
in the mPFC during the task.
In the PCA, only neurons with an average
firing rate over 1 Hz are considered.
Spike trains from 4 seconds prior to and 4 seconds after
each press event are isolated.
The perievent spike train is binned in 1 ms bins and
convolved with a Gaussian filter ($\sigma = $ 25 ms;
PCs are consistent with many other $\sigma$ values).
These spike density functions
are normalized to Z-scores,
and then averaged over all trials to
produce a matrix in which each row is
the normalized average response of a neuron
4 seconds before and after a lever press.
Singular value decomposition
is performed on that matrix
to isolate the principal components,
which are normalized to Z-scores.
The amount of variance accounted for
by each principal component
is computed as the square of the component's eigenvalue
over the sum of all squared eigenvalues
($s_i^2 / \sum s^2$).

As previously described in \citetext{Narayanan2009}
and summarized in Figure~2,
the first two principal components
of the experimental data
are modulated around the lever press
and appear to track
whether the subject has pressed the lever and
the relative amount of time
that the lever has been pressed.
\citeauthor{Narayanan2009} also found that the neural activity
in the mPFC changed significantly depending on
the outcome of the previous trial;
the first two principal components
change significantly after errors,
and appear to contain information that an error occurred.
This information would be necessary
in order to adapt behavior based on the last trial.

Interestingly, in the post-correct case,
the first principal component
is highly correlated with the integral
(i.e., cumulative sum) of the second principal component
during the post-correct trial
(Figure~2E).
Since the first two principal components
account for over half of the variance
in the neural data (Figure~2B),
we hypothesize that the mPFC
represents the task state (PC1)
and the integral of the task state (PC2).
Implementing integration
in a population of neurons has
been widely explored
\cite{Seung1996,Shadlen2001,Wang2002,Eliasmith2005}.
In the post-error case,
these correlations are significantly weaker,
indicating that the integration
taking place in the mPFC
depends on prior outcomes.
The firing patterns of individual neurons,
related to the two leading PCs,
are shown for post-correct trials in Figure~2F.
Nearly equal fractions of neurons
became more or less active at times
when the PCs were maximally positive or negative.
In these plots, neural activity
is synchronized to the start of the trial (time 0).
The cue was presented at 1 second.
The animals were required to respond
to the cue with a reaction time less than 0.6 second,
and feedback about the outcome
was given approximately 0.1 second later.
The range of feedback times are indicated
near the upper right corner of each plot.
Importantly, the trials used for this analysis,
and shown in Figure~2, were all correct trials.
The only difference between trial types
was the outcome of the previous trial
(correct or incorrect).
As in \citetext{Narayanan2008},
neural activity did not track
the type of error that occurred
(premature or late).
Population activity was not analyzed
for these subtypes of errors,
as they were present in unequal fractions of trials
over rats and were much less frequent
than the previously correct trials.

\subsubsection{Behavioral data}

In both the experimental and theoretical studies,
we record five relevant event times:
lever press ($t_p$), lever release ($t_r$), reward delivery ($t_{Rw}$),
cue onset ($t_c$), and house lights extinguishing ($t_{TO}$).
A correct trial is identified by
the sequence lever press, cue onset, lever release, reward delivery;
a premature trial is identified by
lever press, lever release, house lights extinguishing;
and a late trial is identified by
lever press, cue onset, house lights extinguishing.
Trial outcomes are identified
from the overall sequence of events
by filtering for these sequences.
Reaction times are calculated
only for correct trials,
and are defined as the difference between
cue onset and lever release.

\section{Theoretical results}

Here we describe the results of simulating
the dynamical systems in Sections~\ref{sec:dint}
and \ref{sec:control} directly.
This provides a general characterization
of the expected low-dimensional dynamics
of the high-dimensional neural simulation.

\subsection{The double integrator can predict cue time} \label{sec:params}

\begin{table}
  \begin{center}
    \begin{tabular}{ccccc}
      \toprule
      $\beta$ & $R$ & $E$ \\
      \midrule
      0.44 & 2 & 0.5 \\
      \bottomrule
    \end{tabular}
  \end{center}
  \caption{Parameters used to simulate the dynamical system
    described in Equation~\eqref{eq:dint}.
    Appropriate values were found through simulation;
    see Figure~4.}
  \label{tab:sim-param}
\end{table}

The purpose of the double integrator
is to predict the time at which the cue occurs.
Figure~4 shows that this is possible.
The parameter $\beta$ controls how fast
the second dimension of the double integrator state
($x_2$) increases (see Equation~\eqref{eq:dint}),
resulting in different states at $t_c$
(see Figure~4B);
since this parameter is encoded in the strength
of the connection between the two integrators,
this is a natural target for learning
after errors.
Given a particular value for $\beta$,
the state of the system at the time of the cue,
$t_c$, can be predicted given
the initial conditions of the double integrator
at the start of the trial;
values at $t_c$ for $\beta=0.44$ are shown
in Figure~4A.

The state of the system at the start of a trial
(i.e., the initial conditions of the dynamical system)
depend on the outcome of the previous trial.
Equation \eqref{eq:dint} describes how the state changes
based on the two possible outcomes,
reward or light extinguishing.
For sufficiently high values of $R$
(see Figure~4C),
the system starts the next trial
near the origin,
resulting in the system predicting
the time of the cue,
and therefore employing the adaptive strategy.
For sufficiently high values of $E$
(see Figure~4D),
the system starts the next trial near $(-1, -1)$,
resulting in the system being unable to predict
the time of the cue (see Figure~4A),
and therefore employing the cue-responding strategy.
Note that correct and error trials
can be distinguished,
and therefore used to modify behavior
and drive learning;
however, premature trials
and late trials cannot be distinguished
on the next trial,
which is consistent with the results of
\citetext{Narayanan2009}.

Appropriate parameter values were determined
using these simulations
and used for all subsequent simulations,
unless otherwise noted.
The values used are listed in
Table~\ref{tab:sim-param}.

\subsection{The adaptive control network can react faster}

Equation~\eqref{eq:release} defines a signal
that causes the network to release the lever
when the predicted cue approaches.
Figure~5 shows a representative
correct trial from the cue-responding
and adaptive control networks,
as well as a trial in which
the cue was mispredicted,
and the subsequent corrected trials
following the error.
Importantly, Figure~5B
shows the adaptive control network
can release the lever
at the exact time of the cue.

\subsection{The double integrator state can drive learning}

The state of the system reflects the outcome
of the previous trial, and therefore
can affect behavior on the next trial.
The system can also be used to drive
learning in order to maximize long-term reward.
While the proposed system
does not employ learning,
previous work has proposed that the
mPFC drives learning of action outcomes.
Specifically, \citetext{Alexander2011} have proposed
the PRO model, in which the mPFC tracks the predicted values
of action outcomes in order to compute
prediction errors that drive learning.
While the signals tracked
by the double integrator model
exist to modify action timing,
the signals used to drive learning
in PRO can be computed from
the double integrator model as well,
as shown in Figure~6.
The only difference that cannot
be attributed to the task
is that the state is tracked throughout
the trial in the PRO model,
while in the double integrator,
reward turns off the integrator
responsible for tracking task state.
Despite this difference,
the similarity in signals
tracked by the two models
suggests that the double integrator,
which we use to adaptively control action timing,
can also be used to drive learning.

\section{Simulation results}

The above theoretical results suggest that
the underlying low-dimensional dynamics
of the postulated network
should be appropriate
for capturing the qualitative features of mPFC activity.
We now examine
both the qualitative and quantitative fit
of the high-dimensional spiking network
to the recorded empirical data.

\subsection{The spiking double integrator has the same principal components as experimental data}

Figure~7 compares
the two leading principal components
of the spiking implementation of the double integrator network
to those of the experimental data.
The simulated principal components
are very similar to the experimental principal components
\cite{Narayanan2009}:
Pearson $R^2 = 0.82, 0.97, 0.95$
for the first component
and $R^2 = 0.91, 0.81, 0.84$
for the second components in
the post-correct, post-premature,
and post-late trials respectively.

\subsection{A single integrator does not have the same principal components as experimental data}

We also performed principal component analysis
on neural populations representing only one
of the two dimensions tracked by
the double integrator network.
Figure~8
shows the results of the analysis
for the population tracking $x_1$.
While the first principal component
closely matches the experimental data
(Pearson $R^2 = 0.54, 0.86, 0.80$
for post-correct, post-premature, and post-late
trials respectively),
the second principal component
does not (Pearson $R^2 = 0.15, 0.26, 0.30$).
This suggests that there exist
neurons that are sensitive
to both dimensions of the double integrator.

\subsection{Spiking networks can predict the cue}

Figure~9 shows that
the decoded values of spiking neural networks
implementing the cue-responding
and adaptive control networks
closely resemble the signals
directly simulated with
the dynamical system
in correct, premature, and late trials.
Despite the introduction of the oscillation
and the additional injected noise current,
the cue can be predicted,
and the outcome of the previous trial
has a significant effect
on the system state during a trial.
The spiking networks are capable
of the adaptive control described
in the theoretical results,
as can be seen in Figure~9A.

\subsection{Adaptive control network matches observed performance}

Figure~10 shows the performance
of the experimental subjects,
the spiking cue-responding network,
and the full spiking adaptive control network.
The lengths of the simulations were
varied to match the number of trials
performed by subjects.
The mean of the median reaction times
is 272$\pm$48ms for the experimental subjects,
365$\pm$24ms for the cue-responding models,
and 240$\pm$62ms for the adaptive control model.
The mean performance (\% correct trials) is
70.8\% for experimental subjects,
98.6\% for cue-responding models,
and 92.5\% for adaptive control models.
Therefore, only the adaptive control model
has statistically indistinguishable reaction times
compared to the experimental subjects.
The cue-responding network is significantly slower,
but makes fewer errors than the adaptive control network.
Both simulated models make fewer errors
than experimental subjects.

\section{Discussion}

Our results show that a spiking implementation
of the double integrator network
\cite{Singh2006} can reproduce the dynamics
of population activity in the rat mPFC
\cite{Narayanan2009} and can be
used to modify a feedforward control system
in order to adaptively control
the timing of actions.
The ability to predict the time of the cue
produces faster reaction times at the cost
of additional premature errors.

An interesting feature of the model
is that there is significant randomness
involved in model generation (see section~\ref{sec:nef}).
This results in significant variability
in the performance of simulated subjects,
despite all but one performing the task well
(see Figure~10).
Also notable is that the model
does not employ learning:
the weights determined
before the simulation do not change.
The ability to keep track of the state
of the system over time through recurrent connections
is what enables the model to adapt.
By defining the dynamics of other tasks
and implementing them
in recurrently connected neural networks,
we can start to form general theories
of how biological systems can learn
to perform complex tasks.

\subsection{Relation to other models}

In this study, we make
an adaptive control hypothesis
of mPFC function in reaction time tasks.
This hypothesis is compatible
with previous hypotheses suggesting
that the mPFC plays a role
in the regulation of
reinforcement learning parameters.
One such example is the PRO
model by \citetext{Alexander2011}
(though these arguments also
apply to related models,
such as \citenoparens{Khamassi2012}).
The PRO model proposes that the mPFC
is involved in a form of
temporal difference learning.
As shown in Figure~6,
the positive and negative reward prediction error
computed by PRO can also be computed
by the double integrator,
assuming that the value function
learned by the system steadily increases.

In our model, we interpret the steadily increasing
signal as representing time elapsed rather
than value over time;
this means that there is only one parameter
that must be learned
($\beta$, the speed at which $x_2$ increases),
rather than learning the value at every timestep
in the simulation.
In addition to being an easier target for learning,
we further suggest that time tracking
is a more natural interpretation for
such a ramping signal.
Theoretically, it is more likely that
the value signal associated with those
prediction errors is constant
over time rather than steadily increasing
\cite{Niv2008}.
We suggest that the ramping signal
shown in \citetext{Alexander2011} might
reflect the temporal dynamics of the task
instead of serving as a value signal.

The model presented in this study keeps track
of the results of the previous trial,
and therefore can only learn
on a trial-by-trial basis.
Some quantities like volatility,
which is thought to be encoded
in the mPFC \cite{Behrens2007},
would require additional integrative
populations to aggregate information
across many trials.
While we have not presented
such a model in this study,
we believe that a similar dynamical approach
can be used to construct models
that encode volatility
and other quantities
in spiking neural networks,
which can be directly compared
to neural recordings.

\subsection{Adaptation and learning}

A primary difference between
the adaptive control network we have presented
and previous hypotheses of mPFC function
is that we do not employ learning
to change connection weights
during a simulation.
Instead, we achieve adaptive behavior
through recurrently connected
integrative populations that
maintain information across trials.
We believe that this type of adaptability
should be explored in more depth in mPFC models.

We are able to construct a spiking neural network
that performs the simple RT task
because we have made a hypothesis of the
dynamics governing the task.
One issue that we have not attempted to address
in this work is how those dynamics
could be learned from the sensory
information available during the task.
We believe that this learning procedure
is central to understanding the mPFC,
and the brain in general.
However, in order to characterize
this learning procedure,
we must first have examples
of the endpoints of the learning procedure.
We believe that
the adaptive control network,
which uses the double integrator
to predict cue times,
is an example of one possible endpoint.
By understanding the dynamics
of other tasks and how those dynamics
might be implemented in spiking neural networks,
we can begin to form theories of
how spiking neural networks
performing complex tasks might be learned.
In previous work, we have demonstrated
that a spike-based learning rule
can generate a neural integrator
like those used in this study \cite{Macneil2011},
but learning the entire network
described here is the subject of future work.

\subsection{Limitations}

Our model is specific to the simple reaction time task
being performed by the experimental subjects,
and therefore can only be directly
applied to experimental data
of subjects performing this task.
However, we believe that the
methods used to determine the
dynamical system governing the task
and to implement the dynamical system
in a spiking neural network
are general,
and can be used to explain many other tasks.
Determining the dynamics of other tasks
in a dynamical systems framework
can help identify general principles
of neural function.

Even in the context of the simple reaction time task,
our model cannot be directly compared
to all of the experimental data available.
We have chosen to implement the model
in a spiking neural network,
as we believe that it provides the most
direct comparison to
the data recorded from the biological system.
However, because the mappings from
spiking activity to EEG and fMRI recordings
are not well understood,
we cannot directly explain
the wealth of data from humans performing
similar reaction time tasks.
Previous studies of mPFC function
in humans have found evidence
for low-frequency rhythmic activity being
generated when mistakes are made
and on trials after such errors
\cite{Cavanagh2009}.
A recent study reported the same types
of error-encoding cortical rhythms
in rat mPFC \cite{Narayanan2013}
and showed that mPFC activity constrains
rhythmic activity in motor cortex.
It has been suggested that the expression
of these rhythms reflects alterations
in cross-laminar processing
\cite{Carracedo2013}.
These effects could be examined by
using specific models for different
cell types and laminar patterns
for connecting neurons together.

\subsection{Predictions}

Unlike previous mPFC models,
we predict that mPFC inactivation
negatively impacts behavioral performance
of tasks that require precisely timed actions,
in addition to disrupting
the learning of such tasks.

At the neural level,
the spiking implementation
of the network predicts that
neurons in the mPFC will ramp up in activity
over the course of a goal-directed task.
However, we predict that those ramping signals
are an attempt to encode
the temporal dynamics of the task,
not an attempt to encode action-outcome values.
This prediction could be tested
by recording neural ensembles
in the mPFC and motor cortex during learning.
One important study would be to record
the mPFC of naive rats,
as was done in the motor cortex
in \citetext{Laubach2000}.
In that earlier study, a major limitation
was that the rats only experienced
brief foreperiods (400--600ms) before
the trigger stimulus, and ramping activity
was not apparent in the PCA-style analyses
(unpublished observations).
A better test of this idea would be to extend
the required timing of the action to 1 second
and record in the two cortical areas simultaneously.
Neural integration could develop with the increased
length of the foreperiod,
and the rate of integrator ``ramping''
might track the animals' prediction of the cue time.
Alternatively, recordings could be made
in well-trained rats who experience
two foreperiods presented in blocks.
We predict that the rate of integration
would change following block changes,
e.g. as the foreperiod switches from
relatively long to short.
To our knowledge, no studies like these
have been done.

We have shown that neurons
in mPFC are tuned to multiple aspects
of the simple RT task
(see Figures~7 and 8).
Since we find neurons that are sensitive
to both task state ($x_1$) and the
length of time in that state ($x_2$),
we predict that single neurons
would be sensitive to additional
aspects of the task in more complex tasks.
This is due to the mPFC as a whole
tracking the dynamics of that task
in order to modify behavior.
Initial support for this idea
can be found in a recent study
of the mPFC that used
a delayed spatial alternation task
and found similar dynamics
in principal component space across
the mPFC ensembles \cite{Horst2012}.
This prediction could be further tested
by using a task that
has different temporal dynamics
in different trial blocks or task contexts,
and determining if there are neurons
with loadings on many primary components
in a principal component analysis.
One such task was used by
\citetext{Delatour2001}.

\bibliographystyle{namedplus}
\bibliography{jneurosci2013}

\clearpage

\section{Legends}

\paragraph{Figure 1}

The simple reaction time task used in the experiment.
Trials are classified as correct, premature, or late
depending on the time of lever release.
The times at which important events occur
in the trials are labeled.
In order, $t_s$ is the start of the trial,
$t_p$ is when the lever is fully pressed,
$t_c$ is the time of the auditory cue,
$t_{Rw}$ is when reward is delivered,
$t_{TO}$ is when house lights are extinguished,
and $t_{ITI}$ is the start of the intertrial interval.
(A) Schematic of each trial type.
(B) Results of simulating each trial type
according to Equation~\eqref{eq:env}.
(C) Finite state machine tracking
task state in the simulation.

\paragraph{Figure 2}

A summary of the principal component analysis done
in \citetext{Narayanan2009}.
(A) A graphical summary of principal component analysis.
Normalized perievent neural data,
in the form of the Z-scores of instantaneous firing rates,
is organized in a matrix, with each row
being the Z-scored firing rates
of a single neuron on a single trial.
Singular value decomposition is performed on the matrix,
resulting in a matrix such that
the number of columns (time bins) is the same.
The rows are now ordered such that the first row
contains largest eigenvector,
which represents the value when the original
data is projected onto the axes of highest variance.
(B) The fraction of variance explained
by each singular value
when PCA is performed on the post-correct neural data
(top), and the post-error data (bottom).
In both cases, the two eigenvectors with the highest
singular values account for nearly 50\% of variance.
The next two eigenvectors also account for an amount
of variance higher than would be expected when
we linearly interpolate from the smaller
singular values.
(C) A summary of the loadings of the two top
eigenvector on each neuron.
If a neuron were encoding both the first and second
eigenvectors to the same degree,
we would expect a horizontal line.
The varied lines criss-crossing each other
suggest that each individual neuron
has different sensitivities
to the first and second principal components.
(D) The top two principal components
for the post-correct trials (black)
and the post-error trials (red).
(E) (Top) Plotting the cumulative sum (integral)
of the first principal component closely matches
the second principal component, and vice-versa
(Pearson $R^2 = 0.904$ and $R^2 = 0.939$ respectively),
in the post-correct case.
(Bottom) In the post-error case,
the first two principal components are no longer
cumulative sums of each other
(Pearson $R^2 = 0.639$ and $R^2 = 0.676$).
This points to neural integration
as a potential mechanism explaining mPFC activity.
(F) Normalized spike-density functions
for all of the neurons analyzed in the post-correct case,
organized by the loading
(left) on the first principal component, and
(right) on the second principal component.

\paragraph{Figure 3}

Block diagram of the adaptive control network.
(Left) The adaptive control model contains
the double integrator network
from \citetext{Singh2006}.
It has been modified by accepting
both press information ($u$)
and error information ($E \; u_{TO}$)
to drive $x_1$, and
reward information ($R \; u_{Rw}$)
to control the integration
of both $x_1$ and $x_2$.
Lines terminating with circles indicate
negative connections;
lines terminating with arrowheads indicate
positive connections.
The output of the double integrator
is the only component
that interacts with the cue-responding network.
(Middle) The cue-responding network
is able to perform
the simple reaction time task
by responding to the Cue ($u_C$)
provided by the environment
(see the text, Section \ref{sec:cue-responding}).
(Right) Values that are not modeled
with spiking neural populations,
and are therefore considered
signals that are provided by the environment.
The lever accepts the decoded
output of the cue-responding network.

\paragraph{Figure 4}

Time tracking in the double integrator model.
$x_2$ (solid black line) ramps up over time,
enabling the network to predict
the time of the cue.
Time labels indicate important events during each trial.
$t_s$ is the start of the trial,
$t_p$ is when the lever is fully pressed,
$t_c$ is the time of the auditory cue,
$t_{Rw}$ is when reward is delivered,
$t_{TO}$ is when house lights are extinguished,
and $t_{ITI}$ is the start of the intertrial interval.
(A) The state of the system at $t_c$
depends on the initial state of the system
at the start of the trial, $t_p$.
The initial state depends on the outcome
of the previous trial;
after a correct trial
the system should start at $(0, 0)$,
and after a premature or late trial,
the system should start at $(-1, -1)$.
(B) The slope of the ramping signal $x_2$
depends on the parameter $\beta$, which
represents the strength of the connection
between $x_1$ and $x_2$.
An ideal $\beta$ pushes $x_2$ far from
the initial state,
but within the representational range
of a population of spiking neurons,
which is normalized to $(-1, 1)$;
therefore, the system should approach $(1, 1)$
at the time of the cue,
which occurs when $\beta \approx 0.44$.
(C) The system is reset to $(0, 0)$
after a correct trial by reward feedback.
The parameter $R$ controls
the strength of the reward feedback;
the system is properly reset
if $R \ge 2$.
(D) The system is driven near $(-1, -1)$
after an error trial by error feedback.
The parameter $E$ controls
the strength of the error feedback;
the system is driven near $(-1, -1)$
if $E \ge 1$.

\paragraph{Figure 5}

Results of control network simulations.
Time labels indicate important events during each trial.
$t_s$ is the start of the trial,
$t_p$ is when the lever is fully pressed,
$t_c$ is the time of the auditory cue,
$t_{Rw}$ is when reward is delivered,
and $t_{TO}$ is when house lights are extinguished.
(A) The cue-responding network.
The release, $u_r$ is initiated only after cue onset, $t_c$.
(B) The adaptive control network in the ideal case.
The release is initiated such that
the lever (dashed black line) is raised
at the time of cue onset.
(C) A premature release in the adaptive control network.
The release is initiated before the time of cue onset
(premature release is emulated by setting $\beta = 0.94$).
(D) The adaptive control network follows
the cue-responding strategy after error trials,
due to the initial state of the double integrator.

\paragraph{Figure 6}


(Left) Signals tracked by the PRO model \cite{Alexander2011}.
(Right) Signals tracked by or computed from
the double integrator network.
In all but one case, the signals are similar,
and differ mainly in task-related properties
(e.g., the length and magnitude of the reward signal).
In both models, all signals can be represented
as vectors if appropriate for the task.
The one divergent case is the state;
in the PRO model, the state signal
persists after the reward has been delivered,
but in the double integrator network
is turned off by the reward.

\paragraph{Figure 7}

Results of principal component analysis
on the data generated by the double integrator model
for correct presses in (A) post-correct trials,
(B) post-premature trials, and (C) post-late trials.
The top row of panels compares
the top two principal components
of the model to those of the experimental mPFC data.
In all cases, the PCs are significantly similar
(Pearson $R^2 > 0.8$).
The middle row of panels shows normalized
peri-event spike densities
for 174 randomly sampled neurons from the
simulated double integrator model.
The bottom row of panels shows normalized
peri-event spike densities
for all 174 neurons recorded
during the experimental study.
The patterns of sensitivity
compared in the middle and bottom rows
are similar, although the recorded neuron responses
are more variable over time.

\paragraph{Figure 8}

Results of principal component analysis
on the data generated by the integrator
tracking $x_1$ in the double integrator model
for correct presses in (A) post-correct trials,
(B) post-premature trials, and (C) post-late trials.
The top row of panels compares
the top two principal components
of the model to those of the experimental mPFC data.
In all cases, the first PC is significantly similar
(Pearson $R^2 > 0.5$),
but the second PC is not (Pearson $R^2 \le 0.3$).
The bottom row of panels shows normalized
peri-event spike densities
for 174 randomly sampled neurons from the
simulated integrator population.

\paragraph{Figure 9}

(A) Decoded values of representative trials of
the control networks simulated with spiking neurons.
From top to bottom, the trials shown are
a correct trial in the cue-responding network,
a late trial in the cue-responding network,
a correct trial in the adaptive control network,
and a premature trial in the adaptive control network.
(B) The decoded value of the double integrator network
that is embedded in the adaptive control network
on a correct trial following a correct trial.
The system starts ($t_s$) and finishes ($t_{ITI}$)
the trial near the origin point.
(C) The decoded value of the double integrator network
on a correct trial following an error trial.
The system starts near the lower left
portion of the state space,
but since the trial was correct,
finishes near the origin point.

\paragraph{Figure 10}

Behavioral performance of the
experimental subjects and
the cue-responding and adaptive control models.
The top row of panels summarizes the number of
correct, premature, and late trials
for each experimental or simulated subject.
The number of total trials
for the simulations was approximately
matched to the experimental subjects.
The bottom row of panels summarizes the reaction times
on correct trials for each subject.
The dotted black line represents
the mean of all subjects' median reaction times
(272$\pm$48ms for the experimental subjects,
365$\pm$24ms for the cue-responding models,
and 240$\pm$62ms for the adaptive control models).

\end{document}
